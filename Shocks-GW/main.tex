\documentclass{nature}
\usepackage{graphicx}
\usepackage{amsmath}
\usepackage{amssymb}
%\usepackage{natbib}
\usepackage{color}
\usepackage{url}
\usepackage{verbatim}
\usepackage{tabu}
\usepackage{sansmath}


\graphicspath{{images/}{./}}
\newcommand{\apj}{Astrophysical Journal}
\newcommand{\mnras}{Monthly Notices of Royal Astronomical Society}
\newcommand{\apjl}{Astrophysical Journal Letters}
\newcommand{\pasj}{Publications of the Astronomical Society of Japan}
\newcommand{\aap}{Astronomy \& Astrophysics}
\newcommand{\aaps}{Astronomy and Astrophysics Supplement}
\newcommand{\nat}{Nature}
\newcommand{\aj}{Astronomical Journal}
\newcommand{\aplett}{Astrophysics Letters}
\newcommand{\araa}{Annual Review of Astron and Astrophys}



\newcommand{\red}{\textcolor{red}}
\newcommand{\blue}{\textcolor{blue}}
\newcommand{\green}{\textcolor{green}}

\defbibheading{subbibliography}{%
  \section*{}}

\let\cite\autocite

\addbibresource{frb.bib}


\title{Gravitational wave window into the early universe}


% The authors and affiliations lists are created by the python script
% authors.py and the data in author_data.py (in this repository).  Do not edit either
% list here.  Edit script and rerun `python authors.py` on the command line.
% In particular, the author list and author list ordering can be adjusted in
% the authors.py script by modifying a variable.
\author{
    {Ue-Li~Pen}$^{14,2,15}$,
    \&
    {Neil Turok}$^{16}$
    }

\begin{document}

\maketitle

\begin{affiliations}
    \item{Canadian Institute for Theoretical Astrophysics, 60 St George St, Toronto, ON, M5S 3H8, Canada}
    \item{Canadian Institute for Advanced Research, program in
        cosmology and gravitation}
    \item{Dunlap Institute for Astronomy \& Astrophysics, University of Toronto, 50 St George St, Toronto, ON, M5S 3H4, Canada}
    \item{Perimeter Institute, 31 Caroline St, Waterloo, Canada}
\end{affiliations}




% Throughout we need to figure out tense: past tense (burst happened in 2011
% and is over) or present ("the burst has RM=..."). -KM

% Magnetars: starquakes vs flares vs crust disruptions? -KM

\begin{abstract}

We present a new probe of the very early universe through
gravitational waves generated from shock waves.  This
effect is able to constrain the proposed formation of primordial
black holes.

\end{abstract}





\printbibliography[segment=\therefsegment,check=onlynew,heading=subbibliography]


\begin{addendum}
    \item
        %\red{Acknowledge hippo}
        %
        %
        %
        %
        %
        %
        Computations were performed on the GPC supercomputer at the SciNet HPC
        Consortium.
        %
    \item[Author Contributions]
        %
        %
    \item[Competing Interests] The authors declare that they have no competing
        financial interests.
    \item[Correspondence] Correspondence and requests for
        materials should be addressed to U.~P. (email: pen@cita.utoronto.ca).
\end{addendum}


\clearpage
%\input{figures}
\clearpage

%\input{methods}

\printbibliography[segment=\therefsegment,check=onlynew,heading=subbibliography]


\clearpage
%\input{extended_data}
\clearpage


\end{document}


